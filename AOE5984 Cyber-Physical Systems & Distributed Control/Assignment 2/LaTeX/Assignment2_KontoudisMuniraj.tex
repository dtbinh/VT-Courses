\documentclass{beamer}

\mode<presentation> {
\usetheme{Malmoe} 
\usecolortheme{beaver} 
}

\usepackage{graphicx} 
\usepackage{booktabs} 
\usepackage{amsmath}
\usepackage{graphicx}
\usepackage[colorinlistoftodos]{todonotes}
\usepackage{hyperref}
\usepackage{multimedia}
\usepackage{media9}
\usepackage{tikz}
\usetikzlibrary{calc,positioning}
\usepackage{xcolor}
\hypersetup{
    colorlinks=true,       
    linkcolor=blue,          
    citecolor=blue,        
    urlcolor=blue           
}
%----------------------------------------------------------------------------------------
%	TITLE PAGE
%----------------------------------------------------------------------------------------

\title[CPS]{\textcolor{black}{{Consensus and cooperation in networked multi-agent systems \cite{p1}}}} 
\subtitle[]{}

\author{George Kontoudis, DevaPrakash Muniraj}
\institute[VT] 
{Homework 2\\
AOE5984 Cyber-Physical Systems and Distributed Control\\
Spring 2017\\
\medskip
\it{Aerospace and Ocean Engineering Department, Virginia Tech} 
}
\date{\today}

\setbeamertemplate{footline}[text line]{%
  \parbox{\linewidth}{\vspace*{-8pt}\today 
  \hfill\insertshortsubtitle
  \hfill\insertpagenumber}}
\setbeamertemplate{navigation symbols}{}

\begin{document}

\begin{frame}[plain]
\titlepage 
\end{frame}


\begin{frame}
\frametitle{Outline} 
\tableofcontents 
\end{frame}

%----------------------------------------------------------------------------------------
%	PRESENTATION SLIDES
%----------------------------------------------------------------------------------------
%------------------------------------------------
\section{Introduction}
%------------------------------------------------

\begin{frame}
\frametitle{Consensus \& Cooperation}
This paper provides a framework for analysis of consensus algorithms for multi-agent network systems
\begin{itemize}
\item Consensus is defined as reaching an agreement regarding a certain quantity of interest that depends on the state of all agents \vspace{0.2cm}
\item A protocol, also called consensus algorithm, is an interaction rule that specifies the exchange information between an agent and its neighbors on the network\vspace{0.2cm}
\item Networked systems, that are included in agents, are equipped with sensing, computing, and communicating devices
\end{itemize}
\end{frame}

%------------------------------------------------

\begin{frame}
\frametitle{Consensus in Networks}
\begin{itemize}
\item For a directed graph $G=(V,E)$, with a set of nodes $V={1,2,...,n}$ and edges $E \subseteq V \times V$ a simple consensus algorithm of a $n$th order linear system on a graph is 
\begin{equation*}
\dot{x}_i = \sum_{j \in N_i}(x_j(t)-x_i(t))+b_i(t), \hspace{0.2cm} x_i(0)=z_i \in \mathbb{R}, b_i(t)=0
\end{equation*}
with collective dynamics
$\dot{x} = -Lx$
\item Since all row-sums of the Laplacian are zero, L has always a zero eigenvalue $\lambda_1=0$
\item The consensus value is the avg of the initial states $a=\frac{1}{n}\sum_i z_i$
\end{itemize}
\end{frame}

%------------------------------------------------

\begin{frame}
\frametitle{The $f$-Consensus problem \& Cooperation}
Differences between constrained and unconstrained problems
\begin{itemize}
\item In unconstrained problems the state of all agents asymptotically become the same  
\item In constrained problems (f-consensus problems) the state of all agents asymptotically become $f(z)$ 
\end{itemize}
\vspace{0.2cm}
To solve the $f$-consensus problem we need 
\begin{itemize}
\item Cooperation from all agents
\item Willingness to participate from all agents
\end{itemize}
cooperation and willing to participate from all agents
\end{frame}

%------------------------------------------------

\begin{frame}
\frametitle{Applications (1/2)}
Common consensus problems for multi-agent systems
\begin{itemize}
\item Synchronization of coupled oscillators which has dynamics
\begin{equation*}
\dot{\theta}_i = \kappa \sum_{j \in N_i}sin(\theta_j-\theta_i) + \omega_i,
\end{equation*}
where $\omega_i$ is the frequency and $\theta_i$ the phase of the $i$th oscillator
\item Flocking theory of mobile agents with sensing and communication devices, using proximity graphs
\item Fast consensus in small-worlds deals with network design problem. The problem is addressed with either design of weights or design of topology.
\end{itemize}
\end{frame}

%------------------------------------------------

\begin{frame}
\frametitle{Applications (2/2)}
\begin{itemize}
\item Rendezvous in space which reaches a consensus in position by a number of agents
\item Distributed sensor fusion in sensor networks to implement or approximate a Kalman-filter, or estimate linear least-squares
\item Distributed formation control
\end{itemize}
\end{frame}



%------------------------------------------------
\section{Information Consensus in Networked Systems}
%------------------------------------------------

\begin{frame}
\frametitle{Information Consensus in Networked Systems}
\begin{itemize}
\item Consider the dynamics $\dot{x}_i=u_i$ of a graph $G=(V,E)$, that reaches consensus asymptotically
\item The adjacency matrix is $A=[a_{ij}]$, and the set of neighbors $N_i = {j \in V : a_{ij}\ne 0}$
\item A dynamic graph is time-varying $G(t)=(V,E(t))$ with the $A(t)$ and the linear system is a distributed consensus algorithm
\begin{equation*}
\dot{x}_i(t) = \sum_{j \in N-i}a_{ij}(x_j(t)-x_i(t))
\end{equation*}
\item For undirected graphs ($a_{ij}=a_{ji}$) as $t=\infty$ results the avg of the initial states $a=\frac{1}{n}\sum_i x_i(0)$
\end{itemize}
\end{frame}

%------------------------------------------------

\begin{frame}
\frametitle{Laplacian expression}
\begin{itemize}
\item A Laplacian representation of the dynamics is given by $\dot{x} = -Lx$, where $L=D-A$
\item For undirected graphs the Laplcian satisfies the SoS property $x^{\intercal}Lx=\frac{1}{2}\sum_{(i,j)\in E}a_{ij}(x_j-x_i)^2$
\item By setting $\frac{1}{2}x^{\intercal}Lx=\phi(x)$ we get the gradient-descent algorithm $\dot{x}=-\nabla \phi(x)$
\item For an undirected graph the algorithm converges asymptotically for all initial values
\end{itemize}
\end{frame}

%------------------------------------------------
\subsection{Algebraic Connectivity \& Spectral Properties}

\begin{frame}
\frametitle{Algebraic Connectivity \& Spectral Properties}

\end{frame}

%------------------------------------------------
\subsection{Convergence Analysis for Directed Networks}

\begin{frame}
\frametitle{Convergence Analysis for Directed Networks}

\end{frame}

%------------------------------------------------
\subsection{Convergence in Discrete-Time and Matrix Theory}

\begin{frame}
\frametitle{Convergence in Discrete-Time and Matrix Theory}

\end{frame}

%------------------------------------------------
\subsection{Performance of Consensus Algorithms}

\begin{frame}
\frametitle{Performance of Consensus Algorithms}

\end{frame}

%------------------------------------------------
\subsection{Alternative Forms of Consensus Algorithms}

\begin{frame}
\frametitle{Alternative Forms of Consensus Algorithms}

\end{frame}

%------------------------------------------------
\subsection{Weighted-Average Consensus}

\begin{frame}
\frametitle{Weighted-Average Consensus}

\end{frame}

%------------------------------------------------
\subsection{Consensus under Communication Time-Delays}

\begin{frame}
\frametitle{Consensus under Communication Time-Delays}

\end{frame}


%------------------------------------------------
\section{Consensus in Dynamic Networks}
%------------------------------------------------

\begin{frame}
\frametitle{Consensus in Dynamic Networks}


\end{frame}

%------------------------------------------------
\section{Cooperation in Networked Control Systems}
%------------------------------------------------

\begin{frame}
\frametitle{Cooperation in Networked Control Systems}


\end{frame}

%------------------------------------------------
\begin{frame}
\frametitle{Collective Dynamics of Multi-Vehicle Formations}

\end{frame}

%------------------------------------------------
\begin{frame}
\frametitle{Stability of Relative Dynamics of Formations}

\end{frame}

%------------------------------------------------
\section{Simulations}
%------------------------------------------------

\begin{frame}
\frametitle{Consensus in Complex Networks}


\end{frame}

%------------------------------------------------

\begin{frame}
\frametitle{Multi-vehicle Formation Control}


\end{frame}

%------------------------------------------------
\section{References}

\begin{frame}
\frametitle{References}
\footnotesize{
\begin{thebibliography}{99} % Beamer does not support BibTeX so references must be inserted manually as below

\bibitem[Olfati, 2007]{p1} R. Olfati-Saber, J. Alex Fax, R. M. Murray (2007)
\newblock Consensus and cooperation in networked multi-agent systems
\newblock \emph{Proceedings of the IEEE, 95.1 215-233}, 2007.

\end{thebibliography}
}
\end{frame}

%------------------------------------------------
\section{}
\begin{frame}
\begin{center}
\Huge {Thank You!}
\end{center}
\end{frame}

%----------------------------------------------------------------------------------------

\end{document} 